\chapter*{Závěr}
\addcontentsline{toc}{chapter}{Závěr}

V rámci této práce byl naimplementovaný engine pro Hru Dungeon Master. Engine 
obsahuje podporu pro všechny funkce z originální hry. Nicméně všechny konkrétní funkce
nejsou naimplementované, což dává možnost navázat na práci případným následovníkům. 
Pro implementaci byl použit jazyk C\#, platforma .NET a framework MonoGame. 
Většina částí enginu je jednoduše rozšiřitelná či modifikovatelná, což vede 
na dobrou udržitelnost celého systému. Primárně je engine určený pro hru 
Dungeon Master a dokáže si herní úrovně načíst z originální binárních dat. Avšak 
načítání herních úrovní je v oddělené vrstvě, a proto je možné tuto vrstvu upravit kvůli případným 
rozšířením nebo ji je možné nově naimplementovat i pro jiné vstupní formáty.
Z toho důvodu lze tento engine využít i pro implementaci jiných her na podobných principech.
Engine má také oddělenou renderovací vrstvu, což umožňuje jednoduše doimplementovat hře lepší look and feel.
Tento projekt tak může sloužit pro vzdělávání jazyka C\# a objektově orientovaného programování. 
Studenti si tak můžu vyzkoušet do enginu dodělat nové komponenty a tak si vyzkoušet objektově 
orientované programování v praxi.

\section*{Future works}
Na tuto práce je potenciálně možné navázat v následujících bodech:

\begin{itemize}
\item Dodělání všech funkcí Dungeon Masteru, tak aby hra byla hratelná až do konce.
\item Dodělání zobrazovací vrstvy, tak aby odpovídala vzhledu původní hry Dungeon Master.
\item Dodělání lepší 3D zobrazovací vrstvy.
\item Udělat podporu pro textový výstup zobrazovací vrstvy pro možnost automatizovaného kontrolování například systémem CodEx. 
\item Vytvořit editor pro hru Dungeon Master. 
\item Vytvořit jinou hru na tomto enginu.
\item Upravit engine tak, aby nebyl vázaný na stejně velké dlaždice uspořádané do mřížky, a umožňoval tak
	flexibilnější tvorbu herních úrovní. 
\end{itemize}




